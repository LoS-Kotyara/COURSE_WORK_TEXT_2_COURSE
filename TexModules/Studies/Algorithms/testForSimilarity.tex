\subsection{Определение правильного написания слов}

Для определения схожести двух слов была разработана вспомогательная функция, которая возвращает не только расстояние Дамерау-Левенштейна, но и отношение расстояния к длине слов и схожесть слов по полученным данным.

Листинг реализации алгоритма Дамерау-Левенштейна приведён в приложении~\ref{sec:d} на странице~\pageref{lst:damLev}, Листинг~\ref{lst:damLev}. 

Для определения правильности написания слов также была разработана функция, которая получает на вход массив слов (разбитую на слова строку) и возвращает массив объектов, представляющих собой обработанные слова. Структура объектов;

\begin{itemize}
  \item word - исходное слово;
  \item correct - правильно ли написано слово;
  \item suggestions - варианты правильного написания слова, если оно введено неверно;
  \item spellChecked - вспомогательный параметр, который определяет, проверено слово или нет.
\end{itemize}

Suggestions из функции возвращаются не в случайном порядке, а в порядке уменьшения схожести с исходным словом.

Также в теле функции находятся транслитерация слова, его double\_metaphone и русский metaphone. Это делается для того, чтобы произвести максимально тщательный анализ слов. Алгоритм анализа схожести слов:

\begin{enumerate}
  \item Находятся translit, double\_metaphone и cyrMetaphone исходного слова;
  \item Каждое слово словаря, которое различается по длине от исходного не более чем на 1, обрабатывается с исходным словом алгоритмом Дамерау-Левенштейна;
  \item Если расстояние не больше двух либо схожесть слов хотя бы 0.9, слово словаря добавляется в список возможных вариантов правильного написания исходного слова
  \item Полученный массив возможных написаний сортируется по схожести и из функции возвращаются первые 5 возможных написаний.
\end{enumerate}

Листинг реализации этого алгоритма приведён в приложении~\ref{sec:d} на странице~\pageref{lst:similarity}, Листинг~\ref{lst:similarity}.

Этот алгоритм был протестирован на случайных текстах для определения правильности предложенных исправлений и скорости работы программы. Листинг теста и результаты представлены в в приложении~\ref{sec:d} на страницах~\pageref{lst:testProg}, Листинг~\ref{lst:testProg} и \pageref{lst:testRes}, Листинг~\ref{lst:testRes} соответственно.