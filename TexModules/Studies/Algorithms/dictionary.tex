\subsection{Создание словаря}

Перед тем, как реализовать алгоритмы нахождения схожести слов, было необходимо перед этим создать словарь, по которому и будет определяться схожесть слов и предлагаться, если необходимо, варианты замены слов.

Сначала нужно было либо создать, либо скачать готовый сборник русских слов. Я выбрал второй вариант и поэтому скачал с \href{http://speakrus.ru/dict/index.htm}{сайта ``Архивы форума "Говорим по-русски"''} частотный словарь русской литературы.

Затем, с помощью команды \textbf{file -i litc-win.txt} определил исходный формат скачанного файла и с помощью \textbf{iconv -f iso-8859-1 -t UTF-8 -o litc-win.txt litc-win.txt} изменил кодировку на UTF-8.

Затем выполнил преобразования файла, которые удалили из него слова с частотой менее 1, удалил из всех строк значения частот. 

Далее была спроектирована структура словаря, где каждому слову соответствовал объект:

\begin{itemize}
  \item word - само слово;
  \item translit - транслит слова;
  \item metaphone - результат выполнения алгоритма Double Metaphone для транслита слова;
  \item cyrMetaphone - результат выполнения русской адаптации алгоритма Metaphone для слова.
\end{itemize}

Ввиду сложности реализации алгоритма Double Metaphone, реализация алгоритм была добавлена в виде внешнего пакета. Подробнее эту реализацию можно посмотреть по ссылке~\cite{doubleMetaphoneNPM}.

Реализация остальных алгоритмов будет представлена в следующих параграфах.

\subsubsection{Приведение слова к транслиту}

Для привидения слова к транслиту, необходимо сначала определиться, какой системы транслитерации придерживаться. Я выбрал систему транслитерации международных телеграмм~\cite{instukcia} наиболее подходящую для словаря.

Алгоритм приведения слова к транслитерации:

\begin{enumerate}
  \item Получение входного слова;
  \item Создание объекта, содержащего правила транслитерации для каждой буквы;
  \item Инициализация новой стоки newString, в которую будут добавляться буквы транслитерации слова;
  \item Для каждой буквы входного слова:
   \begin{itemize}
    \item Если в объекте есть ключ, соответствующий очередной букве входной строки, то в newString добавляется значение, соответствующее этому ключу;
    \item Если в объекте отсутствует ключ, соответствующий очередной букве входной строки в нижнем регистре, то в newString добавляется эта буква;
    \item Если в объекте есть ключ, соответствующий очередной букве строки в нижнем регистре, то newChar присваивается значение, соответствующее этому ключу в верхнем регистре;
  \end{itemize}
  \item Полученная строка newString подаётся на выход.
\end{enumerate}

Листинг реализации этого алгоритма приведён в приложении~\ref{sec:a} на странице~\pageref{lst:translit}, Листинг~\ref{lst:translit}.

\subsubsection{Получение русского metaphone}

Алгоритм русской адаптации metaphone можно найти в параграфе~\ref{Metaphone}, страница~\pageref{Metaphone}.

Листинг реализации этого алгоритма приведен в приложении~\ref{sec:b} на страница~\pageref{lst:cyrMetaphone}, Листинг~\ref{lst:cyrMetaphone}.

\subsubsection{Получение словаря}

Словарь с предложенной выше структурой можно реализовать с помощью следующего алгоритма:

\begin{enumerate}
  \item Инициализация пустого словаря dict;
  \item Чтение данных из текстового файла с массивом слов словаря и частотой использования этих слов;
  \item Полученные данные разделить на отдельные строки;
  \item В каждой строке удалить частоту использования;
  \item Удалить все слова с длиной менее 1;
  \item Получить транслитерацию, double\_metaphone и русский метафон каждого слова;
  \item Объединить полученные данные для каждого слова и вывести в новый файл.
\end{enumerate}

Листинг реализации этого алгоритма приведён в приложении~\ref{sec:c} на странице~\pageref{lst:dict}, Листинг~\ref{lst:dict}.

Результат работы алгоритма (показаны первые 5 слов из словаря):

\lstinputlisting[caption={Первые 5 слов полученного словаря}]{./TexModules/Appendix/progs/first5words.txt}