\subsection{Metaphone}\label{Metaphone}

Для того, чтобы уменьшить время поиска похожих слов, можно предварительно воспользоваться одним из фонетических алгоритмов.

Фонетические алгоритмы сопоставляют словам с похожим произношением одинаковые коды, что позволяет производить поиск таких слов на основе их фонетического сходства~\cite{phonetic}.

Например, слова `Desert' и `Dessert' будут иметь схожие фонетические коды.

Одним из таких алгоритмов является Metaphone. Этот алгоритм преобразует слова к кодам переменной длины, состоящим только из букв, по сложным правилам. Алгоритм включает в себя следующие шаги~\cite{phonetic2}:

\begin{enumerate}
  \item Удаление повторяющихся соседних букв кроме буквы C;
  \item Если слово начинается с `KN', `GN', `PN', `AE', `WR', убирается первая буква (`KN' -> `N', `GN' -> `N', ...);
  \item Опускается `B' в `MB', если `MB' - суффикс;
  \item `C' преобразуется в `X', если за ним следует `IA' или `H' (только если он не является частью `-SCH-', в этом случае он преобразуется в `K'). `C' преобразуется в `S', если за ним следуют `I', `E' или `Y'. В остальных случаях `С' преобразуется в `К';
  \item `D' преобразуется в `J', если за ним следуют `GE', `GY' или `GI'. В противном случае `D' преобразуется в `T';
  \item Удаляется `G', если за ним следует `H', причем `H' стоит не в конце слова и не перед гласным. Также удаляется `G', если за ним следует `N' или `NED', и он является окончанием;
  \item `G' преобразуется в `J', если до `I', `E' или `Y', и это не в `GG'. В противном случае `G' преобразуется в `K';
  \item Опускается `H', если он стоит после гласного и не перед гласным;
  \item `CK' преобразуется в `K';
  \item `PH' преобразуется в `F';
  \item `Q' преобразуется в `K';
  \item `S' преобразуется в `X', если за ним следуют `H', `IO' или `IA';
  \item `T' преобразуется в `X', если за ним следует `IA' или `IO'. `TH' преобразуется в `0'. `T' опускается, если за ним следует `CH';
  \item `V' преобразуется в `F';
  \item `WH' преобразуется в `W', если он стоит в начале. `W' опускается, если за ним не следует гласная;
  \item «X» преобразуется в «S», если он стоит в начале. В противном случае «X» преобразуется в «KS»;
  \item `Y' опускается, если за ним не следует гласная;
  \item `Z' преобразуется в `S';
  \item Опускаются все гласные, кроме начального.
\end{enumerate}

В последствие алгоритм Metaphone был улучшен, была выпущена вторая версия алгоритма,которая получила название Double Metaphone, в которой, в отличие от первой версии, применимой только к английскому языку, учитывалось происхождение слов, особенности их произношения~\cite{phonetic2}. Для таких слов результатом работы являются два кода - основной вариант произношения и альтернативный~\cite{phonetic}. Алгоритм Double Metaphone сложнее, чем его предшественника, увидеть его можно в статье Лоуренса Филипса `The double metaphone search algorithm' \url{https://dl.acm.org/doi/10.5555/349124.349132}.

Пример работы алгоритма: `My String' будет преобразовано к `MSTRNK'.

Алгоритм Metaphone был адаптирован к русскому языку. Для русского языка алгоритм состоит из пяти шагов~\cite{phonetic2}:

\begin{enumerate}
  \item Преобразование гласных путем следующих подстановок: О, Ы, Я -> А; Ю -> У; Е, Ё, Э, ЙО, ЙЕ -> И;
  \item Оглушение согласный букв, за которыми следует любая согласная, кроме Л, М, Н или Р, либо согласных на конце слова путем следующих подстановок: Б -> П; З -> С; Д -> Т; В -> Ф; Г -> К;
  \item Удаление повторяющихся букв;
  \item Преобразование суффикса слова путем следующих подстановок: УК, ЮК -> 0; ИНА -> 1; ИК, ЕК -> 2; НКО -> 3; ОВ, ЕВ, ИЕВ, ЕЕВ -> 4; ЫХ, ИХ -> 5; АЯ -> 6; ЫЙ, ИЙ -> 7; ИН -> 8; ОВА, ЕВА, ИЕВА, ЕЕВА -> 9; ОВСКИЙ -> @; ЕВСКИЙ -> \#; ОВСКАЯ -> \$; ЕВСКАЯ -> \%;
  \item Удаление букв Ъ, Ь и дефиса.
\end{enumerate}

Из-за небольшого числа правил, адаптированный для русского языка алгоритм Metaphone не отождествляет некоторые схожие фонетически слова.