\subsection{Расстояние Левенштейна}
Рассмотрим следующую задачу: имеется две строки s1 и s2, необходимо перевести либо s1 в s2, либо s2 в s1, используя операции:

\begin{itemize}
	\item i: Вставка символа в произвольное место;
	\item d: Удаление символа с произвольной позиции;
	\item r: замена символа на другой.
\end{itemize}
Расстоянием Левенштейна для перевода s1 в s2 для рассматриваемой задачи будет d(s1, s2) - минимальное количество операций i/d/r для перевода s1 в s2, а редакционное предписание - перечисление операций для перевода с их параметрами.

Для расстояния Левенштейна справедливы следующие утверждения:
\begin{enumerate}
	\item $d(s1, s2) \geq \|s1\| - \|s2\|$
	\item $d(s1, s2) \leq \max(|s1|, |s2|)$
	\item $d(s1, s2) = 0 \iff s1 = s2$
\end{enumerate},
где |s| - это длина строки s.

Искомое расстояние формируется через вспомогательную функцию D(m, n), находящую редакционное расстояние для срезов s1[0, m] и s2[0, n], где D(i,~j) находится по формуле~\ref{eq:1}


\[
D(i, j) = \begin{cases}
0 & ; i = 0, j = 0 \\
i & ; j = 0, i > 0 \\
j & ; i = 0, j > 0 \\
D(i - 1, j - 1) & ; s1[i] = s2[j] \\
\min\begin{pmatrix}
D(i, j - 1) \\
D(i - 1, j)\\
D(i - 1, j - 1)
\end{pmatrix} + 1 & : j > 0, i > 0, s1[i] \neq s2[j]
\end{cases} \label{eq:1} \tag{1}
\]

, где (i, j) - клетка матрицы, в которой мы находимся на данном шаге.~\cite{redRasst}

Здесь $ D(i, 0) =  $ и $ D(0, j) = j $ получены из соображения, что любая строка может получиться из пустой, добавлением нужного количества нужных символов, любые другие операции будут только увеличивать оценку.

В общем случае имеем D(i,~j) по формуле~\ref{eq:2}~\cite{redRasst}
\[
D(i, j) = D(i - 1, j - 1), если s1[i] = s2[j],
\label{eq:2} \tag{2}
\]
иначе по формуле~\ref{eq:3}
\[
D(i, j) =\min\begin{pmatrix}
D(i, j - 1) \\
D(i - 1, j)\\
D(i - 1, j - 1)
\end{pmatrix} + 1
\label{eq:3} \tag{3}
\].

В данном случае мы выбираем наиболее выгодную операцию: удаление символа (D(i - 1, j)), добавление (D(i, j - 1)) или замена (D(i - 1, j - 1)).

Вычисление матрицы D можно произвести, используя следующий псевдокод:

\begin{lstlisting}
input: strings a[1..length(a)], b[1..length(b)]
output: distance, integer

let d[0..length(a), 0..length(b)] be a 2-d array of integers, dimensions length(a)+1, length(b)+1
// note that d is zero-indexed, while a and b are one-indexed.

for i := 0 to length(a) inclusive do
	d[i, 0] := i
for j := 0 to length(b) inclusive do
	d[0, j] := j

for i := 1 to length(a) inclusive do
	for j := 1 to length(b) inclusive do
		if a[i] = b[j] then
			cost := 0
		else
			cost := 1
		d[i, j] := minimum(d[i-1, j] + 1,     // deletion
						   d[i, j-1] + 1,     // insertion
						   d[i-1, j-1] + cost)  // substitution
return d[length(a), length(b)]
\end{lstlisting}

Результат работы алгоритма для слов ``Пять'' и ``Семь'' представлен в таблице~\ref{tab:1}.

%% \newpage

\begin{table}[H]
	\caption{Пример работы алгоритма}
	\label{tab:1}
	\begin{tabular}{|l|l|l|l|l|l|}
		\hline
		&   & П & Я & Т & Ь \\ \hline
		& 0 & 1 & 2 & 3 & 4 \\ \hline
		С & 1 & 1 & 2 & 3 & 4 \\ \hline
		Е & 2 & 2 & 2 & 3 & 4 \\ \hline
		М & 3 & 3 & 3 & 3 & 4 \\ \hline
		Ь & 4 & 4 & 4 & 4 & 3 \\ \hline
	\end{tabular}
\end{table}