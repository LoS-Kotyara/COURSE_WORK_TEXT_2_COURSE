\section{Средства для программной реализации}

Для написания программы, которая бы  реализовала алгоритмы, описанные выше, нужно, в первую очередь, выбрать подход, которого мы будем придерживаться при разработке. Есть два подхода: разработка нативных приложений, т.е. приложений, которые работают только на определённой платформе или на определённом устройстве, и разработка кроссплатформенных приложений, т.е. приложений, которые способны работать с двумя и более платформами.

В следствие чего был выбран кроссплатформенный подход, т.к. он позволяет разрабатывать программу на одной операционной системе и знать, что разработанная программа будет работать и на других ОС, необходимо лишь перекомпилировать приложение для нужной платформы.

Выбирал я между двумя фреймворками языков, Qt (C++) и Electron (JS), с помощью которых можно писать кроссплатформенные приложения, и, для того чтобы определиться, какой фреймворк использовать, необходимо рассмотреть несколько критериев:

\begin{itemize}
  \item Лёгкость поддержки приложения;
  \item Легкость разработки приложения;
  \item Распространённость фреймворка.
\end{itemize}

В итоге Electron, оказался наиболее подходящим для разработки и поддержания.

\subsection{Electron}

Electron - это фреймворк для разработки настольных кроссплатформенных приложений с использованием HTML, CSS и JS~\cite{electron}. Его особенность состоит в том, что если ты знаешь как разрабатывать сайты, например, то ты сможешь разработать и настольное приложение. По сути, приложение, написанное на Electron представляет собой окно браузера, в котором открыто единственное окно --- ваше приложение.

Процесс разработки на Electron разбит на две взаимно зависимые части: разработка интерфейса приложения (фронтэнда) и разработку логической части приложения (бекэнда). Обе эти части можно написать используя лишь JS, HTML и CSS.

\subsection{HTML}

HTML - это стандартизированный язык гипертекстовой разметки документов во Всемирной паутине. Браузер может интерпретировать описанный с помощью HTML документ и отобразить его структуру на экране пользователя.

После загрузки веб-страницы, браузер создаёт DOM - Document Object Model - объектную модель документа этой страницы. Благодаря этой модели, содержимое сайта можно прочитать и изменить с помощью скриптов, в частности, с помощью JS.
Благодаря этому можно описать данные в виде набора утверждений и формул, изменение которых ведет к автоматическому перерасчёту всех зависимостей, сделать сайт реактивным с помощью, например, JS.

Во многих фронтэнд-фреймворках реализована реактивность. Я выбрал VueJS как один из самых популярных и прогрессивных фреймворков.

\subsection{Vue}

VueJS  позволяет декларативно отображать данные в DOM с помощью простых шаблонов. Например, следующий пример кода создаст в DOM компонент, содержащий приветствие:

\begin{lstlisting}[language=JavaScript]
<div id = "app">
  {{message}}
</div>

var app = new Vue ({
  el: "#app",
  data: {
    message: "Hello"
  }
})
\end{lstlisting}

Данные и DOM теперь реактивно связаны - при изменении данных, DOM автоматически перестроится.

В Vue каждое поле данных автоматически разбивается на пары геттер и сеттер. С их помощью Vue может следить, какие данные читались или изменялись и может определить, какие факторы влияют на отрисовку отображения.

Каждому экземпляру компонента приставлен связанный с ним экземпляр наблюдателя, который помечает все поля, затронутые при отрисовке, как зависимые. Когда вызывается сеттер поля, помеченного как зависимость, этот сеттер уведомляет наблюдателя, который, в свою очередь, инициирует повторную отрисовку компонента~\cite{vue}.

С помощью этого можно разрабатывать крупные проекты, не отвлекаясь на проблему синхронизации данных.


