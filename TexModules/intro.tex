\intro
%% Вступление (цели)
Нечёткий поиск - это способ поиск информации, которая совпадает шаблону сравнения приблизительно или очень близкого шаблону значения. Алгоритмы нечёткого поиска применяются для распознавания текста, например, при занесении информации с отсканированных документов в базу, нахождения произошедших от некоторого слова слов, в поисковых системах, проверки орфографии и других областях~\cite{obzorMetodov}.

Проблема нечеткого поиска текстовой информации может заключаться в следующем: имеется некоторый текст. Пользователь вводит в поле поиска запрос, представляющий из себя некоторое слово или последовательность слов, для которых необходимо найти в тексте все совпадения с запросом с учетом всех возможных допустимых различий. Например, при запросе "polynomial" нужно найти также слово "exponential".

Для оценки сходства двух слов в тексте используются специальные метрики нечеткого поиска, которые определяются как минимальное количество односимвольных операций (вставки, удаления, замены), необходимых для превращения одной строки в другую. В качестве метрик используются сходство Джардо-Винклера, расстояние Хемминга, расстояния Левенштейна и Дамерау-Левенштейна и другие.

Целью курсовой работы является реализация алгоритма нахождения расстояния Дамерау-Левенштейна для двух слов, поиск возможности минимизировать время поиска схожих слов и использование полученных алгоритмов в приложении.

Для достижения цели необходимо решить следующие задачи:

\begin{enumerate}
  \item Реализовать алгоритмы нахождения схожести двух слов;
  \item Создать словарь слов, по которому будет производиться поиск схожих слов;
  \item Разработать приложение, текстовый редактор, использующий эти методы и словарь.
\end{enumerate}